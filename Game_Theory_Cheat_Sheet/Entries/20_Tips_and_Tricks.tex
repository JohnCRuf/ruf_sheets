\section{Tips and Tricks}

\subsection{Solution Concept Selection Guide}
\begin{center}
\begin{tabular}{@{}p{3.8cm}|p{4.5cm}@{}}
\textbf{Game Structure} & \textbf{Solution Concept} \\ \hline
Simultaneous, complete info & NE (pure or mixed) \\
Simultaneous, want to rule out dominated play & Rationalizability / ISD \\
Correlation device available & Correlated Equilibrium \\
Two-player zero-sum & Maxmin = Minimax (LP) \\
Sequential, complete info, perfect info & SPE via backward induction \\
Sequential, complete info, imperfect info & SPE via OSD principle \\
Static, incomplete info (types) & Bayesian NE \\
Sequential + incomplete info & Sequential Equilibrium \\
Robustness to trembles & Trembling Hand Perfect \\
Mechanism / auction design & Revelation principle $\to$ direct mechanism 
\end{tabular}
\end{center}

\textbf{Key refinement hierarchy:} Trembling Hand $\subseteq$ Sequential Eq.\ $\subseteq$ SPE $\subseteq$ NE. Each successive concept eliminates more profiles by imposing rationality at more information sets with consistent beliefs.

%%% Decision Theory and Strategic Games %%%
\subsection{Properties of payoff functions}
Let $\alpha_1, \alpha_2$ represent two mixed strategies, then note the following properties of utility functions:
\begin{align*}
    u(\alpha_1,\alpha_2)&=\sum_{a_2 \in A_2} \alpha_2(a_2) u(\alpha_1,a_2) \\ &=\sum_{a_1 \in A_1} \alpha_1(a_1) u(a_1,\alpha_2) \\
    &=\sum_{a_1 \in A_1} \sum_{a_2 \in A_2} \alpha_1(a_1) \alpha_2(a_2) u(a_1,a_2)
\end{align*}
\subsection{Correlated Equilibrium Tricks}
Key deviation for verifying CE: fix $\hat{a}_i, \hat{a}_i' \in A_i$. The CE condition reduces to:
$$\sum_{a_{-i}} \mu(\hat{a}_i, a_{-i}) (u_i(\hat{a}_i,a_{-i})-u_i(\hat{a}_i',a_{-i})) \geq 0$$
for all $i$, $\hat{a}_i$, $\hat{a}_i'$. Check this inequality for each action pair.

\subsection{Tricks with TPZSG's}
Recall that, if \(u_1\) represents the utility of the maximizer, we can always write \(u_2(a_1,\,a_2) = -u_1(a_1,\,a_2)\). 

%%% Games of Incomplete Information %%%
\subsection{Sequential Equilibria}

\begin{itemize}
    \item If an assessment \((\beta,\, \sigma)\) is consistent, it is sequentially rational if and only if it has no one-shot deviations. Here, we understand an OSD to be a strategy \(\sigma'_i\) which differs from \(\sigma_i\) at exactly one information set.
\end{itemize}

%%% Worked Examples %%%
\subsection{Example: Mixed NE in 2\texttimes2}
\begin{center}
\begin{tabular}{c|cc}
 & $L$ & $R$ \\ \hline
$T$ & $2,1$ & $0,0$ \\
$B$ & $0,0$ & $1,2$
\end{tabular}
\end{center}
Pure NE: $(T,L)$ and $(B,R)$. For mixed NE, let P1 play $T$ w.p.\ $p$ and P2 play $L$ w.p.\ $q$. P2 indifferent $\implies p \cdot 1 = (1-p)\cdot 2 \implies p = 2/3$. P1 indifferent $\implies q\cdot 2=(1-q)\cdot 1\implies q=1/3$. Mixed NE: $(\frac{2}{3}T + \frac{1}{3}B,\, \frac{1}{3}L + \frac{2}{3}R)$.

\subsection{Example: Backward Induction}
P1 chooses $\{L,R\}$; then P2 observes and chooses $\{l,r\}$. Payoffs $(u_1,u_2)$: $(L,l)\to(2,1)$, $(L,r)\to(0,0)$, $(R,l)\to(1,2)$, $(R,r)\to(3,0)$. \\
\textbf{Step 1}: At P2's nodes: after $L$, P2 picks $l$ ($1>0$); after $R$, P2 picks $l$ ($2>0$). \\
\textbf{Step 2}: P1 anticipates: $L\to(2,1)$, $R\to(1,2)$. P1 picks $L$. \\
BI outcome: $(L, (l,l))$ with payoffs $(2,1)$.

\subsection{Example: BNE (Cournot)}
Firm 1: $c_1=0$. Firm 2: $c_2 \in \{0,2\}$ w.p.\ $\frac{1}{2}$ each. $P=10-Q$. Firm 2's BR: $q_2(c_2)= \frac{10-c_2-q_1}{2}$. Firm 1 maximizes $\mathbb{E}[q_1(10-q_1-q_2)]$, giving $q_1^* = \frac{10 - \frac{1}{2}q_2(0) - \frac{1}{2}q_2(2)}{2}$. Substituting BRs and solving: $q_1^* = \frac{11}{3}$, $q_2(0)=\frac{19}{6}$, $q_2(2)=\frac{13}{6}$.

\subsection{Example: Verifying a CE}
In the Battle of the Sexes with payoffs:
\begin{center}
\begin{tabular}{c|cc}
 & $O$ & $F$ \\ \hline
$O$ & $3,2$ & $0,0$ \\
$F$ & $0,0$ & $2,3$
\end{tabular}
\end{center}
Claim: $\mu(O,O)=\frac{1}{2},\, \mu(F,F)=\frac{1}{2}$ is a CE. Check: P1 told $O$: deviation to $F$ gives $\mu(O|O)(u_1(O,O)-u_1(F,O))= 1\cdot(3-0)=3 \geq 0$ \checkmark. P1 told $F$: $1\cdot(2-0)=2\geq 0$ \checkmark. By symmetry, P2's conditions hold too.
