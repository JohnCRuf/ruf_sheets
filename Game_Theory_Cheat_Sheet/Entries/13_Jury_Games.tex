\section{Lecture 13: Jury Games}
\subsection{Jury Game Set-up}
\begin{itemize}
    \item $\Theta= \{Innocent, Guilty \}$
    \item $A_i= \{Convict, Acquit$
\end{itemize}

\begin{align*}
     u_i(a, \theta)=\begin{cases} 
      -(1-y) & \text{if } \theta=G \text{ and } |\{i|a_i=c\}|< k \\
      -y & \text{if } \theta=I \text{ and } |\{i|a_i=c\}|\geq k \\
      0 & \text{o.w}
   \end{cases}
\end{align*}
for some $k \in N=\{1,...,n\}$ and $y \in (0,1)$.

Where y is the threshold probability of guilt s.t. conviction is preferred.
\subsection{Jury Game: Odd Majority Rule}
If every juror is truthful, then I only consider my decision when it is pivotal. IE: $\{\#j\neq i |a_j=c\}=\frac{n-1}{2}$. Thus the posterior of guilt is:

\begin{align*}
    \pi(\theta=G|t_i=G, \text{pivotal}) &= \\ \frac{\frac{1}{2}x*x^{\frac{n-1}{2}}(1-x)^{\frac{n-1}{2}}}{\frac{1}{2}x*x^{\frac{n-1}{2}}(1-x)^{\frac{n-1}{2}}+\frac{1}{2}(1-x)*(1-x)^{\frac{n-1}{2}}(x)^{\frac{n-1}{2}}} \\
    &=x \\
    \pi(\theta=I|t_i=I, \text{pivotal})
\end{align*}
\subsection{Jury Game: Unanimity Equilibrium}
We are looking for a setup where:

\begin{align*}
    \sigma_i(c|G)=1 \\
    \sigma_i(c|I)= \gamma \ \in (0,1)
\end{align*}
This implies:
\begin{align*}
    y&=\pi(G|t_i=I, \ \text{pivotal}) \\
    &=\frac{\frac{1}{2}(1-x)(x+(1-x)\gamma)^{n-1}}{\frac{1}{2}(1-x)(x+(1-x)\gamma)^{n-1}+\frac{1}{2}x(1-x+x\gamma)^{n-1}}
\end{align*}
This comes out to:
\begin{align*}
    y=\frac{x z^{\frac{1}{n-1}}-(1-x)}{x-(1-x)z^{\frac{1}{n-1}}}
\end{align*}
Where $z=(\frac{1-x}{x} \frac{1-y}{y})^{\frac{1}{n-1}}$.

Notice that, under unanimity, truthful voting is \textbf{not} an equilibrium. In fact, the equilibrium under unanimity requires mixing when a juror receives an innocent signal.
\subsection{Jury Game: Even Majority Rule}
In this case, an agent is pivotal only when $k-1=N / 2$ of the other agents vote to convict and the remaining $N / 2$ - 1 other agents vote to acquit. In this event, using Bayes' rule, the posterior probability that the defendant is guilty conditional on a signal of $t_{i}=G$ is
\begin{align*}
    \pi(G \mid G)&=\frac{(1 / 2) x^{(N / 2+1)}(1-x)^{(N / 2-1)}}{(1 / 2) x^{(N / 2+1)}(1-x)^{(N / 2-1)}+(1 / 2) x^{(N / 2-1)}(1-x)^{(N / 2+1)}} \\ &=\frac{x^{2}}{x^{2}+(1-x)^{2}}
\end{align*}
Note that
$$
\begin{array}{r}
x<\frac{x^{2}}{x^{2}+(1-x)^{2}} \\
\Longleftrightarrow x>x^{2}+(1-x)^{2}
\end{array}
$$
$$
\begin{aligned}
&\Longleftrightarrow \frac{1}{2}<x<1
\end{aligned}
$$
As $x \in(1 / 2,1)$ and $x \geqslant y \geqslant 1-x$ the agent will convict when they observe $t_{i}=G$. When they observe $t_{i}=I$ their posterior is
$$
\pi(I \mid I)=\frac{(1 / 2) x^{(N / 2)}(1-x)^{(N / 2)}}{(1 / 2) x^{(N / 2)}(1-x)^{(N / 2)}+(1 / 2) x^{(N / 2)}(1-x)^{(N / 2)}}=\frac{1}{2}
$$
Thus $\pi(G \mid I)=1 / 2$ and the agent will only acquit if $y<\frac{1}{2}$.
