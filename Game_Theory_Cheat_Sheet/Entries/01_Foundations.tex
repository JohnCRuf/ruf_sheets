\section{Basics of Game Theory}
\subsection{Common Proof Techniques}
Want to show that $A \implies B$. \\
\textbf{Contradiction}
Suppose C is a known false statement
\begin{align*}
   \neg (A \implies B) \implies C
\end{align*}
\textbf{Contraposition}
\begin{align*}
   \neg A \implies \neg B
\end{align*}
\textbf{Exhaustion}
Suppose $A = \{A_1,A_2...,A_n\}$
\begin{align*}
    A_1 &\implies B \\
    A_2 &\implies B \\
    &... \\
    A_n &\implies B
\end{align*}
\subsection{What is a game?}
Game has 3 items:
\begin{itemize}
    \item Players
    \item Actions for each player
    \item Preferences for each player over everyone's actions
\end{itemize}

\subsection{Properties of preferences}
A preference relation is a method of ranking alternatives in a set X. There are 3 kinds of relations:
\begin{itemize}
    \item Preference: $\succeq$
    \item Indifference: $x \sim y$ if $x \succeq y$ \& $y \succeq x$
    \item Strict Preference $x \succ y$ if $\neg (y \succeq x)$
\end{itemize}

A relation, R, can have the following properties: \\
\textbf{Completeness}
\begin{align*}
    \text{if} \  \forall x,y \in X: \ \text{Either} \ (x R y), (y R x), \ \text{or both}
\end{align*}
\textbf{Transitive}
\begin{align*}
    \text{if} \  \forall x,y,z \in X: \\
    (x R y) \ \& \ (y Rz) \implies (x R z)
\end{align*}
\textbf{Rational}
\begin{align*}
    \text{If \textbf{Complete} \& \textbf{Transitive}}
\end{align*}
\textbf{Reflexive}
\begin{align*}
    x R x \  \forall \ x \in X
\end{align*}
\textbf{Asymmetric}
\begin{align*}
    x R y \implies \neg(y R x)
\end{align*}
\textbf{Anti-symmetric}
\begin{align*}
    xRy \ \& \ yRx \implies x=y
\end{align*}
\textbf{Negative Transitive}
\begin{align*}
    \text{if} \  \forall x,y,z \in X: \\
    \neg (x R y) \ \& \ \neg (y R z) \implies \neg (x R z)
\end{align*}
\textbf{Total}
\begin{align*}
    xRy \ \text{or} \ yRx \ \text{whenever} \ x \neq y 
\end{align*}
\subsection{Implications of Rationality}
If a preference relation $\succeq$ is rational, then :

\begin{itemize}
    \item The indifference relation $\sim$ is reflexive and symmetric
    \item The strict preference relation $\succ$ is asymmetric and negative transitive
    \item If $\succeq$ is complete and anti-symmetric then $\succ$ is total
\end{itemize}
Propositions about utility and preferences

\textbf{Proposition: If $\succeq$ is represented by U then $\succeq$ is rational.} \\
Note: The converse does not hold in general.

\subsection{Rational implies U counterexample}
Let $X=[0,1]$ and we have lexicographic preferences, i.e.\ $(x,y) \succeq (x',y')$ if $x \geq x'$ or $x=x'$ and $y \geq y'$. By contradiction, a utility function existing would imply that the cardinality of $[0,1]$ is weakly less than the rationals. 

\subsection{Useful Math Definitions}
\textbf{Surjective}: A function $f: A \rightarrow B$ is surjective if $\forall b \in B \exists a \in A$ s.t. $f(a)=b$ \\
\textbf{Injective}: A function $f: A \rightarrow B$ is injective if $\forall a \in A$, $f(a) \neq f(a')$ whenever $a\neq a'$. \\
\textbf{Countable}: A set X is countable if $\exists$ a surjective function f: $\mathbb{N} \rightarrow X$

\subsection{Countability, Rationality, and Utility Functions}
If a set \(X\) is countable, and \(\succsim\) is a rational preference relation on \(X\), then there exists some utility function \(u\) which represents \(\succsim\).

\subsection{Separability and Utility Functions}
If \((X,\,d)\) is a separable metric space (equipped with the metric \(d\)), \(\succsim\) is rational, and for all \(x \in X\) the set \(\{y \in X\,|\, x \succ y\}\) is open, then there exists some utility function which represents \(\succsim\).
