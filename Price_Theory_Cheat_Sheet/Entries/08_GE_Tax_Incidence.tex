\section{General Equilibrium \& Tax Incidence}

\subsection{Edgeworth Box}
Two consumers ($A$, $B$), two goods, fixed endowments $\omega$.
\begin{itemize}[nosep]
    \item \textbf{Feasible:} $x^A + x^B = \omega^A + \omega^B$
    \item \textbf{Pareto optimal:} MRS$^A$ = MRS$^B$ (contract curve)
    \item \textbf{Competitive eq.:} Utility maximization on budget sets at prices $p$, markets clear
\end{itemize}
1st Welfare Thm: competitive eq.\ is Pareto optimal. 2nd Welfare Thm: any Pareto optimum is a competitive eq.\ with lump-sum transfers.

\textit{Application:} To check whether trade improves on autarky, verify MRS$^A \neq$ MRS$^B$ at the endowment point. If they differ, gains from trade exist.

\subsection{\texorpdfstring{2 $\times$ 2}{2x2} GE Model (Jones, 1965)}
Two goods ($X$, $Y$), two factors ($K$, $L$), CRS, competitive markets. Hat notation: $\hat z = dz/z$.

\textbf{Zero-profit:} $a_{LX}w + a_{KX}r = p_X$, \quad $a_{LY}w + a_{KY}r = p_Y$.

\textbf{Full-employment:} $a_{LX}X + a_{LY}Y = L$, \quad $a_{KX}X + a_{KY}Y = K$.

\textbf{Stolper-Samuelson (price $\to$ factor returns):}
$\hat w > \hat p_X > \hat p_Y > \hat r$ when $X$ is $L$-intensive.

Factor returns are \textit{magnified}: the factor used intensively gains more than the price increase; the other factor loses.

\textbf{Rybczynski (endowment $\to$ output):}
$\hat X > \hat L > 0 > \hat Y$ when $X$ is $L$-intensive.

\textit{Application (trade):} A country abundant in $L$ exports $X$ (the $L$-intensive good). Opening trade raises $p_X \implies$ raises $w$ and lowers $r$ (Stolper-Samuelson). Predicts that trade liberalization hurts the scarce factor.

\textit{Application (immigration):} An increase in $L$ expands the $L$-intensive sector and contracts the $K$-intensive sector (Rybczynski). Output adjusts, but factor prices remain unchanged if both goods are produced (factor price insensitivity).

\subsection{Harberger Tax Incidence}
Tax on capital in sector $X$ at rate $t$. Capital earns $r_X = r + t$ in $X$, $r_Y = r$ in $Y$ (mobile capital, net return $r$).

\textbf{Symmetric Cobb-Douglas case:} If both sectors have identical factor shares and $\sigma = 1$: capital bears 100\% of the tax: $dr/dt = -1$.

\textbf{Intuition:} Tax drives $K$ from $X$ to $Y$, lowering $r$ everywhere. With symmetric technologies, GE adjustment exactly offsets PE incidence.

\textbf{General case:} Who bears the tax depends on: (1) factor intensities, (2) $\sigma$ in each sector, (3) product demand elasticities. PE incidence can be misleading---a tax appearing to fall on consumers may fall entirely on a factor in GE.

\textit{Modeling tip:} In Harberger models, there are 4 unknowns ($\hat w$, $\hat r$, $\hat X$, $\hat Y$) and 4 equations (two zero-profit, two full-employment). Log-linearize and solve the $4\times 4$ system.
