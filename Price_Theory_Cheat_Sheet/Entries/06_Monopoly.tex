\section{Monopoly \& Market Power}

\subsection{Static Monopoly}
$\max_q \pi = P(q)q - C(q)$. FOC:
$$MR = P + P'q = MC \qquad \iff \qquad P\!\left(1 + \frac{1}{\epsilon_D}\right) = MC$$
\textbf{Lerner index:} $\frac{P - MC}{P} = -\frac{1}{\epsilon_D}$. More elastic demand $\implies$ smaller markup.

\textit{Linear demand} $P = \alpha - \beta q$: $q^* = \frac{\alpha - MC}{2\beta}$, $P^* = \frac{\alpha + MC}{2}$.

$DWL = \frac{(\alpha - MC)^2}{8\beta}$ \quad (one quarter of competitive surplus is lost).

\subsection{Two-Period Monopoly with Copying}
Inventor sells in period 1; each unit spawns $n$ copies in period 2. Inverse demand $v(c)$, cost $w/A$.

\textbf{Interior:} When $n$ is small, the two-period FOC collapses to the static monopoly result---the inventor fully captures period-2 surplus through period-1 pricing.

\textbf{Corner} ($c_{i2}=0$, linear demand):
$$c_{i1}^* = \frac{(n+1)(\alpha - w/A)}{(n(n+1)+2)\beta}, \qquad c_2^* = nc_{i1}^*$$
As $n\to\infty$: $c_{i1}^*\to 0$, $c_2^*\to 2c^*_{\text{comp}}$. Perfect copying destroys creation incentives but yields competitive allocation.

\textit{Application (IP policy):} $n$ parameterizes enforcement. Weak IP (high $n$) $\implies$ more consumption but less creation. The optimal $n$ trades off static efficiency against dynamic incentives.

\subsection{Price Discrimination}
\textbf{1st degree:} Charge WTP. Efficient but extracts all $CS$.

\textbf{2nd degree:} Menu of bundles; distort low-type quantity to prevent high-type mimicking. Self-selection via IC constraints.

\textbf{3rd degree:} Segment markets. FOC per market: $P_i(1 + 1/\epsilon_i) = MC$. Higher price where demand is less elastic.

\textit{Application:} Airline pricing charges business travelers (inelastic) more and leisure travelers (elastic) less. Formally: $P_{bus}/P_{lei} = (1+1/\epsilon_{lei})/(1+1/\epsilon_{bus}) > 1$ when $|\epsilon_{bus}| < |\epsilon_{lei}|$.

\subsection{Price Controls on Monopoly}
A \textbf{price ceiling below monopoly price} can \textit{increase} output: the monopolist faces a kinked demand curve (horizontal at the ceiling, then original demand). If ceiling is between $MC$ and $P^M$: output rises toward competitive level.

\textbf{Quality deterioration:} Under price controls, firms reduce quality until effective price matches unconstrained equilibrium.
