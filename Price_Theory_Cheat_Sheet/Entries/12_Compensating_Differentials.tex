\section{Compensating Differentials \& Location}
When markets are competitive and agents mobile, price differences across goods, jobs, or locations must reflect real differences in attributes---otherwise arbitrage would eliminate them.

\subsection{Hedonic Pricing}
Goods differ along attributes $z = (z_1,\ldots,z_n)$. Price function $P(z)$ in equilibrium:
$$\frac{\partial P}{\partial z_k} = \text{MRS}_{z_k,m} = MC_{z_k}$$
Hedonic gradient identifies the \textit{marginal} buyer's WTP (not average). Full demand identification requires instruments (Rosen's second step).

\subsection{Compensating Wage Differentials}
Workers sort across jobs with different amenities. In equilibrium:
$$w(a) = w^* - v(a)$$
where $v(a)$ is the monetary value of amenity $a$ to the marginal worker.

\textit{Application (VSL):} Jobs with higher fatality risk $\Delta p$ pay compensating differentials $\Delta w$. The value of a statistical life: $VSL = \Delta w/\Delta p$. If a job with 1/10,000 extra death risk pays \$800 more: $VSL = \$8M$.

\subsection{Rent Gradient}
Workers commute to CBD at cost $t\cdot d$. Spatial equilibrium equalizes utility:
$$R(d) = R(0) - t\cdot d$$
Land at city edge: $R(\bar d) = R_a$ (agricultural rent) $\implies$ $R(0) = R_a + t\bar d$.

$\bar d$ increases with population and productivity; decreases with $t$.

\textbf{With heterogeneous wages:} High-wage workers value time more $\implies$ $R'(d) = -w$ (savings from living closer = wage $\times$ commute time saved). Higher $w \implies$ steeper gradient $\implies$ \textbf{convex} rent gradient across worker types. High earners live near CBD.

\textit{Application:} A productivity shock raises CBD wages. Workers bid up nearby rents until $R(d)$ adjusts to equalize utility. Landowners, not workers, capture location-specific surplus.

\textit{Application (Uber):} Lower commuting cost $t$ flattens the rent gradient: city-center rents fall, suburban rents rise, city expands ($\bar d \uparrow$).

\subsection{Spatial Equilibrium}
$V(w_j, r_j, a_j) = \bar V$ across all locations $j$. Higher wages offset by higher rents or worse amenities. \textbf{Cannot infer welfare from wages alone.}
