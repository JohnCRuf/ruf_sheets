\section{Intertemporal \& Exhaustible Resources}

\subsection{Savings \& Borrowing}
Two-period model: income $(Y_1, Y_2)$, consumption $(C_1, C_2)$, interest rate $r$.
$$\max U(C_1, C_2) \quad \text{s.t.} \quad C_1 + \frac{C_2}{1+r} = Y_1 + \frac{Y_2}{1+r}$$
FOC: $\frac{U'_1(C_1)}{U'_2(C_2)} = 1+r$ (MRS $=$ price ratio).

\textbf{Interest rate increases:}
\begin{itemize}[nosep]
  \item \textbf{Saver} ($S_1>0$): substitution $\implies C_1\downarrow$, income $\implies$ can afford more of both. $C_2$ rises unambiguously; $C_1$ ambiguous (but saving increases)
  \item \textbf{Borrower} ($S_1<0$): both effects reduce $C_1$, borrowing falls
\end{itemize}

\subsection{Life-Cycle / Euler Equation}
With time preference $\rho$: $U'(C_t) = \frac{1+r}{1+\rho}U'(C_{t+1})$.
\begin{itemize}[nosep]
  \item $r > \rho$: consumption rises over time (patient agent)
  \item $r < \rho$: consumption falls (impatient agent)
  \item $r = \rho$: flat consumption path
\end{itemize}

\textit{Application (retirement):} Consumption drops at retirement because the \textbf{price of leisure falls} ($w \to 0$), causing substitution toward time-intensive goods and reducing market expenditure.

\subsection{Tree-Cutting / Wine-Selling}
Tree value $W(t)$ grows over time. Optimal harvest: $W'(t^*)/W(t^*) = r$.

\textbf{Rule:} Cut when the \textbf{growth rate of value equals the interest rate}. Before $t^*$: tree grows faster than $r$ (keep). After $t^*$: capital earns more invested elsewhere.

Equivalently: $P'(t)/P(t) = r$ for wine. Hold asset until capital gains rate $=$ opportunity cost.

\subsection{Hotelling Rule (Exhaustible Resources)}
Resource stock $S$, extraction $q_t$, price $P_t$, zero extraction cost.
$$P_t = P_0(1+r)^t$$
\textbf{Price rises at the rate of interest.} If it rose faster, owners delay extraction (excess supply today); if slower, extract now (no future supply).

\textbf{With extraction cost $c$:} $(P_t - c)$ rises at rate $r$. The \textit{rent} (price minus cost) grows at $r$, not the price.

\textbf{Monopolist vs.\ competitive:} With \textit{constant elasticity} demand, the monopolist's extraction path is \textit{identical} to competitive (markups cancel). With linear demand, monopolist conserves more (slower extraction).

\textit{Application:} OPEC behavior. If oil prices are expected to rise faster than $r$, owners restrict supply. Changes in $r$ (e.g., from monetary policy) affect extraction incentives.

\subsection{Investment \& Present Value}
Project with cash flows $\{C_t\}$: $NPV = \sum_t C_t/(1+r)^t$.

\textbf{Investment rule:} Accept iff $NPV > 0$, equivalently iff \textbf{internal rate of return} $> r$.

\textit{Application (PS5 2019 --- light bulb durability):} Competitive firms choose bulb lifespan $T$ to maximize $NPV$ of bulb sales. Optimal $T$: marginal cost of durability $= PV$ of one period's rental value. A monopolist chooses \textit{shorter} lifespan to increase replacement demand (if commitment is possible; cf.\ Coase conjecture otherwise).

\subsection{Capital Tax: Short Run vs.\ Long Run}
SR: capital supply inelastic $\implies$ tax falls on capital (after-tax return $\downarrow$).

LR: capital supply perfectly elastic at $R^*$ (empirically $\approx 6$--$7\%$). Tax is fully passed to labor via higher pre-tax return $\implies$ higher output prices $\implies$ lower real wages.

\textbf{Optimal LR capital tax is zero} (with CRS): eliminating the capital tax raises labor income by \textit{more} than the lost revenue (the DWL triangle is recaptured). Fund the lost revenue with a labor tax instead.
