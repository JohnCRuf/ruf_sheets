\section{The Chicago Approach}

\subsection{Core Methodology}
Price theory is \textbf{applied}: write a model, solve it, check if predictions match data. Use the \textbf{simplest model} that captures the economic force. The goal is not generality but \textit{insight}---a two-good, two-period model that delivers a clear comparative static beats a general framework that delivers ambiguity.

\textbf{Three functions of prices:} (1) transmit scarcity information, (2) provide incentives to economize, (3) direct resources to highest-valued uses. When prices are prevented from adjusting, \textit{something else} clears the market (queues, search, quality degradation).

\subsection{Demand Without Rationality}
\textbf{Becker (1962):} Random choice on a budget line $\implies$ aggregate demand is downward-sloping, HD0, satisfies adding up and Slutsky symmetry. \textit{Implication:} many demand properties come from the budget constraint alone. If a model only needs a demand curve, rationality is not required---budget constraints suffice.

\subsection{Equilibrium Thinking Template}
Never stop at the direct effect:
\begin{enumerate}[nosep]
    \item Write down direct/partial equilibrium effect
    \item Identify margins of adjustment (other prices, quantities, behavior)
    \item Re-solve for new equilibrium; compare with PE
    \item Check: does direct effect survive? Reverse? Amplify?
\end{enumerate}

\textit{Ethanol:} PE: corn demand $\uparrow$. GE: feed prices $\uparrow$, benefits producers $>$ subsidy (multiplier). \textit{Paxlovid:} PE saves lives; GE: less prevention $\implies$ net ambiguous. \textit{Rent control:} PE: rents $\downarrow$; GE: quality $\downarrow$, supply $\downarrow$, misallocation.

\subsection{TFU Toolkit}
Recurring exam patterns:
\begin{itemize}[nosep]
    \item \textbf{Stock vs.\ flow:} asset price $\neq$ rental price; tax equivalence holds only in SS
    \item \textbf{Average vs.\ marginal:} $MC < AC \implies AC$ falling; monopolist targets marginal consumer
    \item \textbf{Pecuniary vs.\ real:} price changes redistribute but are \textit{not} externalities
    \item \textbf{SR vs.\ LR:} stock fixed in SR; adjustment via entry/exit
    \item \textbf{Envelope theorem:} eliminates first-order behavioral adjustments
\end{itemize}
