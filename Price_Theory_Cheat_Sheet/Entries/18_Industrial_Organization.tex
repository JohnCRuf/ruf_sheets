\section{Industrial Organization}

\subsection{Dominant Firm (Price Leadership)}
One large firm (share $S$), competitive fringe. Dominant firm faces \textbf{residual demand}: $D_R = D - S_f$.

\textbf{Residual demand elasticity:}
$$\epsilon^* = \frac{\epsilon_D}{S} - \frac{(1-S)}{S}\epsilon_S^{\text{fringe}}$$
Markup: $\frac{P-MC}{P} = -\frac{1}{\epsilon^*}$. Market power is \textbf{highly nonlinear} in share:

\begin{center}
\begin{tabular}{@{}c|cccc@{}}
Share $S$ & 10\% & 30\% & 50\% & 100\% \\ \hline
Markup & 1.8\% & 5.3\% & 14.3\% & 100\% \\
\end{tabular}
\end{center}
(With $\epsilon_D = -1$, $\epsilon_S^f = 1$.) A firm with 10\% share has negligible pricing power.

\textit{Application (AT\&T/Time Warner):} Vertical merger gives control of content. Market power depends on whether rivals can access substitutes---fringe supply elasticity matters more than market share.

\textit{Worked example:} Market demand $P = 100 - Q$, fringe supply $Q_f = P$, dominant firm $MC = 20$; share $S \approx 50\%$. Residual demand: $Q_R = (100-P) - P = 100 - 2P$, so $P = 50 - Q_R/2$. $MR_R = 50 - Q_R$. Set $MR_R = 20 \implies Q_R = 30$, $P = 35$, fringe supplies 35, total $Q = 65$. Markup: $(35-20)/35 \approx 43\%$.

\subsection{Collusion}
\textbf{Stigler's conditions} (collusion harder when):
\begin{itemize}[nosep]
  \item Many sellers (harder to monitor)
  \item Heterogeneous products or costs
  \item Large, infrequent orders (incentive to cheat)
  \item Low barriers to entry (profits attract entrants)
\end{itemize}

\textbf{Cheating incentive:} Deviating firms earn more per unit (free-ride on collusive price without restricting output).

\textbf{Long-run:} Entry reduces shares $\implies \epsilon^*$ rises $\implies$ markup falls $\implies$ collusion collapses.

\subsection{Vertical Integration}
\textbf{Double marginalization:} Upstream monopolist charges $P_U > MC_U$; downstream monopolist adds its own markup. Final price has \textit{two} markups $\implies$ higher than integrated monopoly price.

\textbf{Vertical integration eliminates one markup:} integrated firm sets $MR = MC_U + MC_D$. Output rises, price falls, \textbf{both firms and consumers gain}. This is why vertical mergers are often pro-competitive.

\textbf{Cournot effect for complements:} $n$ independently priced complements: each monopolist ignores effect of own price on others' demand. Integration internalizes complementarity $\implies$ lower bundle price.

\textit{Application:} Vertical merger eliminates double marginalization on content licensing.

\subsection{Network Effects \& Social Multiplier}
Demand depends on others' consumption: $x_i = f(p, \bar{x})$ where $\bar{x}$ is aggregate.

\textbf{Social multiplier:} If $\partial x_i/\partial\bar{x} = \phi$, the equilibrium response to a price change is:
$$\frac{dx}{dp}\bigg|_{\text{eq}} = \frac{\partial x/\partial p}{1 - \phi}$$
With $\phi > 0$ (conformity), the equilibrium response exceeds the individual response by factor $1/(1-\phi)$.

\textbf{Multiple equilibria} possible with S-shaped adoption curves; small interventions can tip between them.

\subsection{Advertising (Dorfman-Steiner)}
$$\frac{A}{PQ} = \frac{\epsilon_A}{|\epsilon_D|}$$
Advertise more when advertising elasticity is high relative to demand elasticity.

\subsection{Price Discrimination: Welfare}
\textbf{3rd degree:} Welfare effect is ambiguous. Deadweight loss falls in elastic-demand market (output expands) but rises in inelastic market (output falls). \textbf{Total output must increase} for PD to improve welfare.

\textbf{Spence quality distortion:} A monopolist designs quality for the \textit{marginal} consumer, not the average. If marginal consumer values quality less than average, quality is underprovided. If more (e.g., attracting new customers), quality is overprovided.
