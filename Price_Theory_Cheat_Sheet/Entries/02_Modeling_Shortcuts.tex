\section{Modeling Shortcuts \& Setup}

\subsection{Standard Simplifying Assumptions}
Unless contradicted by the problem or making it trivial:
\begin{enumerate}[nosep]
    \item Assume \textbf{normal goods}
    \item \textbf{Ignore income effects} (quasilinear utility)
    \item Assume \textbf{continuously differentiable} utility
    \item $\lim_{x\to 0} U'(x) = \infty$, $\lim_{x\to\infty} U'(x) = 0$ (interior solutions)
    \item Assume \textbf{constant marginal costs}
    \item \textbf{DRS} at firm level, \textbf{CRS} at industry level
    \item \textbf{Representative agent} or ex-ante identical agents
    \item i.i.d.\ shocks; ignore integer constraints
    \item Use only \textbf{two goods} and/or \textbf{composite commodities}
\end{enumerate}

\subsection{Model Setup Checklist}
\textbf{1.\ Count equations vs.\ unknowns.} Adding a constraint without a variable $\implies$ over-determined. Introduce a new margin (search, rationing, wait time).

\textbf{2.\ Choose the right numeraire.} Set $p_m = 1$ or normalize a factor price. All results hold in relative prices.

\textbf{3.\ Representative agent + market clearing.} Normalize to one consumer. Aggregate = individual in equilibrium. Avoids tracking distributions when heterogeneity isn't the focus.

\textbf{4.\ Linear production} $F=AH$. Price pinned to $p = w/A$ by zero profit. Reduces to a pure consumer problem.

\textbf{5.\ One-dimensional heterogeneity.} Index types by $\theta_i$. Optimal behavior monotone in $\theta \implies$ cutoff strategies: $\bar\theta$ separates actions. Equilibrium reduces to finding $\bar\theta$.
